\documentclass[12pt,a4paper]{article}
\usepackage[utf8]{inputenc}
\usepackage[magyar]{babel}
\usepackage[T1]{fontenc}
\usepackage{graphicx}
\usepackage{epstopdf}
\usepackage{caption}
\usepackage{subcaption}
\usepackage{xcolor}
\usepackage{amsmath}
\usepackage{amsfonts}
\usepackage{amssymb}
\usepackage{multirow}
\usepackage{hyperref}
\usepackage[noadjust]{cite}
\usepackage[all]{hypcap}
\usepackage[hypcap]{caption}
\usepackage{braket}
\usepackage{url}
\usepackage{float}
\usepackage{listings}
\setlength{\textwidth}{17cm}
\setlength{\textheight}{24cm}
\setlength{\topmargin}{-2cm}
\setlength{\footskip}{1cm}
\setlength{\evensidemargin}{-0.3cm}
\setlength{\oddsidemargin}{-0.3cm}
\setlength{\parindent}{0cm}

\hypersetup{
    colorlinks=true,
    linkcolor=blue,
    linktoc=all,
    filecolor=magenta,      
    urlcolor=cyan,
    pdfstartview=FitB,
}
 
\urlstyle{same}

\title{\Huge{\textbf{15. Kvantumradír}}}
\author{\textsl{Barna Zsombor, Rédl Anna}}
\date{Mérés időpontja: 2021.02.23.\\
	\textsl{Keddi csoport}\\}
\begin{document}

\begin{titlepage}
\pagenumbering{gobble}
\maketitle
\end{titlepage}

\section{Bevezetés}
\pagenumbering{arabic}
\setcounter{page}{1}


\section{A mérési összeállítás}


\section{Rövid elméleti összefoglaló}


\section{Számolási feladatok}

\subsection{A lézernyalábok által bezárt szög számítása Mach-Zender elrendezésben}

Ha a négy tükör nem tökéletesen párhuzamos, a lencsén áthaladva a nyalábok a fókuszsíkban nem egyetlen pontban metszik egymást, hanem kettő pontban ( $S'$ és $S''$). 

Ha feltételezzük, hogy a nyalábok $\alpha$ szöge nagyon kicsi, akkor az $S'$ és $S''$ pontok, valamint a lencse fősíkjának az $S'S''$ szakasz felezőjével szemközti pontja által definiált egyenlő szárú háromszög szárai $\approx f$, ahol $f$ a lencse fókusztávolsága. 

Ebből a cosinus tétel alapján az $S'S''$ szakasz $d$ hosszúsága az alábbiak szerint becsülhető:

\begin{equation}
d^2\approx 2f^2(1-\cos\alpha)
\end{equation}

$\cos{\alpha}$ -t másodrendig sorbafejtve:

\begin{equation*}
    \cos{\alpha}\approx 1-\frac{\alpha^2}{2}+\mathcal{O}(\alpha^4)
\end{equation*}

Ezt az előző kifejezésbe behelyettesítve:

\begin{equation*}
    d=f\alpha
\end{equation*}

A $d$ szakasz hosszára kaphatunk egy másik egyenletet is az alaphán, hogy az $S'$ és $S''$ pontokból gömbhullámok indulnak ki. A $\theta_m$ irányok, az $m$ rendek, a $d$ réstávolság és a $\lambda$ hullámhossz között az alábbi összefüggés írható fel:

\begin{equation}
d\sin\theta_m=m\lambda
\end{equation}

Ha $\theta_m$ kicsi és a lencse és az ernyő $L$ távolságára igaz, hogy $L\gg f$, akkor élhetünk az alábbi közelítéssekkel a $\theta_m$ irányra és az adott interferenciacsíknak a középső csíktól való $l_m$ távolságára vonatkozóan:

\begin{equation*}
\theta \approx \sin\theta\approx\tan\theta\approx\frac{l_m}{L}
\end{equation*}

Ez alapján az egyenlet a következő formát ölti:

\begin{equation*}
d \approx \frac{m\lambda L}{l_m}     
\end{equation*}

Ebbe behelyettesítve a $d$-re kapott előző kifejezést, a nyalábok párhuzamosságának $\alpha$ hibájára a következő kifejezés adódik: 

\begin{equation}\label{eq: alpha}
\alpha\approx  \frac{m\lambda L}{fl_m}
\end{equation}

\section{Mérési adatok}

\subsection{Jelölt és jelöletlen útvonal}

\begin{figure}[h]
\centering
\begin{minipage}{.5\textwidth}
  \centering
  \includegraphics[width=.9\linewidth]{1.png}
  \captionof{figure}{Jelöletlen útvonal}
  \label{fig:test1}
\end{minipage}%
\begin{minipage}{.5\textwidth}
  \centering
  \includegraphics[width=.9\linewidth]{2.png}
  \captionof{figure}{Jelölt útvonal}
  \label{fig:test2}
\end{minipage}
\end{figure}

\subsection{A kvantumradírozás szemléltetése}
\begin{figure}[h]
\centering
\includegraphics[width=8cm]{3.png}
\captionof{figure}{Bekapcsolt radírozó polárszűrő}
\end{figure}

\pagebreak
\subsection{Egyik ernyőn nincs jel}

\begin{figure}[h]
\centering
\begin{minipage}{.5\textwidth}
  \centering
  \includegraphics[width=.9\linewidth]{4.png}
  \captionof{figure}{Másik ernyőn nincs interferencia}
  \label{fig:test3}
\end{minipage}%
\begin{minipage}{.5\textwidth}
  \centering
  \includegraphics[width=.9\linewidth]{5.png}
  \captionof{figure}{Másik ernyőn van interferencia}
  \label{fig:test4}
\end{minipage}
\end{figure}


\section{A mérés kiértékelése}


\begin{thebibliography}{1}

\bibitem{leiras}
Koltai János: Kvantumradír
(\href{http://wigner.elte.hu/koltai/labor/parts/modern15.pdf}{Mérési leírás})


\end{thebibliography}

\end{document}
